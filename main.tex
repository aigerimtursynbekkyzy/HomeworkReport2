\documentclass[12pt, a4paper]{article}
\usepackage[T1]{fontenc}

\usepackage[english]{babel}
\usepackage{microtype}
\usepackage{amsmath,amsfonts,amsthm}
\usepackage{graphicx}
\usepackage{url}
\usepackage{geometry}
\usepackage{hyperref}
\usepackage{fancyhdr}
\usepackage{enumitem}
\usepackage{tabularx}
\usepackage{mathtools}
\usepackage{csquotes}
\usepackage[style=apa]{biblatex}
\addbibresource{ref.bib}
% Adjust margins here

\geometry{left=3cm, right=3cm, top=3cm, bottom=3cm, headheight=15pt}
\addtolength{\topmargin}{-2.5pt}
% Increase bottom margin to lower page numbering

\pagestyle{fancy}
\fancyhf{} % clear all header and footer fields
\fancyhead[L]{MATH 323: Actuarial Mathematics I  } % left header
\fancyhead[R]{Homework Report 2} % right header
\fancyfoot[C]{\thepage} % center footer
\renewcommand{\headrulewidth}{0.4pt} % header rule width
\renewcommand{\footrulewidth}{0.4pt} % footer rule width

\begin{document}

\begin{titlepage}
    \centering
    
    \vspace*{0.5cm}
    
    {\Large\bfseries MATH 323: Actuarial Mathematics I\par}
    
    \vspace{1cm}
    
    {\large Homework Report 2\par}
    
    \vspace{0.5cm}
    
    {\today\par}
    
    \vspace{1pt}
    
    \includegraphics[width=0.3\textwidth]{NU-logo.png}\\
    \includegraphics[width=0.15\textwidth]{sosah-logo.png}

    \vspace{0.5cm}
    
    Submitted for {\bf MATH 323: Actuarial Mathematics I} at the School of Sciences and Humanities, Department of Mathematics, Nazarbayev University
    
    \vspace{0.5cm}
    
    {\large Student Name:\par}
    \begin{itemize}[leftmargin=5cm,rightmargin=4cm]
        \item  Aigerim Tursynbekkyzy - ID: 202043550
    \end{itemize}

    \vspace{0.5cm}
    
    \flushleft{
  Subject Area: {\bf Theory of Interest} \\
  Description: {\bf Homework Problems in Chapter 4 and Chapter 5 \\
  Course Instructor : {\bf Dongming Wei} \\
    }
    
    \vspace{0.5cm}
    
    {\footnotesize In submitting this work we are indicating
    that we have read the University's Academic Integrity Policy. We
    declare that all material in this assessment is our own work except
    where there is clear acknowledgment and reference to the work of
    others.\par}
\end{titlepage}
The following is the used as solutions samples for each problem:
\newpage
\section*{Problems}
\subsection*{Question 4, Section 2, Chapter 4  (\cite{toi3rd})}

\noindent An annuity-immediate that pays 400 quarterly for the next 10 years costs \$10{,}000. Calculate the nominal interest rate convertible monthly earned by this investment.

\subsection*{Question 7, Section 3, Chapter 4  (\cite{toi3rd})}

\noindent Find an expression for the present value of an annuity-due of \$600 per annum payable semiannually for 10 years if \( d^{(12)} = 0.09. \)

\subsection*{Question 9, Section 3, Chapter 4  (\cite{toi3rd})}

\noindent Find an expression for the present value of an annuity on which payments are \$100 per quarter for five years, just before the first payment is made, if \( \delta = 0.08. \)

\subsection*{Question 10, Section 3, Chapter 4  (\cite{toi3rd})}

\noindent Find an expression for the present value of an annuity on which payments are 1 at the beginning of each 4-month period for 12 years, assuming a rate of interest per 3-month period.

\subsection*{Question 13, Section 4, Chapter 4  (\cite{toi3rd})}

\noindent A sum of \$10{,}000 is used to buy a deferred perpetuity-due paying \$500 every six months forever. Find an expression for the deferred period expressed as a function of \( d. \)

\subsection*{Question 14, Section 4, Chapter 4  (\cite{toi3rd})}

\noindent If \( 3 a_{\angl{n}}^{(2)} = 2 a_{\angl{2n}}^{(2)} = 45 s_{\angl{n}}^{(2)} \), find \( i. \)

\subsection*{Question 17, Section 5, Chapter 4  (\cite{toi3rd})}

\noindent There is \$40{,}000 in a fund which is accumulating at 4\% per annum convertible continuously. If money is withdrawn continuously at the rate of \$2{,}400 per annum, how long will the fund last?

\subsection*{Question 19, Section 5, Chapter 4  (\cite{toi3rd})}

\noindent Find an expression for \( \bar{a}_{\angl{n}} \) if \( \delta_t = \frac{1}{1 + t} \) for \( 0 \le t \le n. \)

\subsection*{Question 22, Section 6, Chapter 4  (\cite{toi3rd})}

\noindent Simplify
\[
\sum_{t=1}^{20} (t + 5)v^t.
\]

\subsection*{Question 27, Section 6, Chapter 4  (\cite{toi3rd})}

\noindent An annuity-immediate has semiannual payments of 800, 750, 700, \dots, 350, at \( i^{(2)} = 0.16. \)  
If \( a_{\angl{10}, 0.08} = A \), find the present value of the annuity in terms of \( A. \)

\subsection*{Question 30, Section 7, Chapter 4  (\cite{toi3rd})}

\noindent Annual deposits are made into a fund at the beginning of each year for 10 years. The first 5 deposits are \$1000 each and deposits increase by 5\% per year thereafter.  If the fund earns 8\% effective, find the accumulated value at the end of 10 years. Answer to the nearest dollar.

\subsection*{Question 31, Section 7, Chapter 4  (\cite{toi3rd})}

\noindent A perpetuity makes payments starting five years from today. The first payment is \$1000 and each payment thereafter increases by \( k\% \) per year.  The present value of this perpetuity is equal to \$4096 when computed at \( i = 25\% \). Find \( k \).

\subsection*{Question 32, Section 7, Chapter 4  (\cite{toi3rd})}

\noindent An employee currently is aged 40, earns \$40{,}000 per year, and expects to receive 3\% annual raises at the end of each year for the next 25 years. The employee decides to contribute 4\% of annual salary at the beginning of each year for the next 25 years into a retirement plan. How much will be available for retirement at age 65 if the fund can earn a 5\% effective rate of interest? Answer to the nearest dollar.

\subsection*{Question 33, Section 7, Chapter 4  (\cite{toi3rd})}

\noindent A series of payments is made at the beginning of each year for 20 years with the first payment being \$100. Each subsequent payment through the tenth year increases by 5\% from the previous payment.  After the tenth payment, each payment decreases by 5\% from the previous payment. Calculate the present value of these payments at the time the first payment is made using an annual effective rate of 7\%.  Answer to the nearest dollar.

\subsection*{Question 34, Section 8, Chapter 4  (\cite{toi3rd})}

\noindent Derive formula (4.40).

\subsection*{Question 36, Section 8, Chapter 4  (\cite{toi3rd})}

\noindent Show that the present value of a perpetuity on which payments are 1 at the end of the 5th and 6th years, 2 at the end of the 7th and 8th years,  
3 at the end of the 9th and 10th years, and so on, is 
\[
\frac{v^4}{i - v d}.
\]

\subsection*{Question 38, Section 8, Chapter 4  (\cite{toi3rd})}

\noindent A perpetuity provides payments every six months starting today. The first payment is 1 and each payment is 3\% greater than the immediately preceding payment. Find the present value of the perpetuity if the effective rate of interest is 8\% per annum.

\subsection*{Question 42, Section 9, Chapter 4  (\cite{toi3rd})}

\textbf{(a)} Find an integral expression for \( (\bar{D} \, \bar{a})_{\angl{n}}. \)

\textbf{(b)} Find an expression not involving integrals for \( (\bar{D} \, \bar{a})_{\angl{n}}. \)

\subsection*{Question 43, Section 9, Chapter 4  (\cite{toi3rd})}

\noindent A one-year deferred continuous varying annuity is payable for 13 years. The rate of payment at time \( t \) is \( t^2 - 1 \) per annum, and the force of interest at time \( t \) is \( (1 + t)^{-1} \). Find the present value of the annuity.

\subsection*{Question 46, Section 9, Chapter 4  (\cite{toi3rd})}

\noindent A family wishes to provide an annuity of \$100 at the end of each month to their daughter now entering college. The annuity will be paid for only nine months each year for four years. Show that the present value one month before the first payment is
\[
1200 \, \ddot{a}_{\angl{4}} \, q^{(12)}_{9/12}.
\]

\subsection*{Question 6, Section 2, Chapter 5  (\cite{toi3rd})}

\noindent A loan of 1 was originally scheduled to be repaid by 25 equal annual payments at the end of each year. An extra payment \( K \) with each of the 6th through the 10th scheduled payments will be sufficient to repay the loan 5 years earlier than under the original schedule. Show that
\[
K = \frac{a_{\angl{20}} - a_{\angl{15}}}{a_{\angl{25}} a_{\angl{5}}}.
\]

\subsection*{Question 7, Section 2, Chapter 5  (\cite{toi3rd})}

\noindent A husband and wife buy a new home and take out a \$150{,}000 mortgage loan with level annual payments at the end of each year for 15 years on which the effective rate of interest is equal to 6.5\%. At the end of 5 years they decide to make a major addition to the house and want to borrow an additional \$80{,}000 to finance the new construction. They also wish to lengthen the overall length of the loan by 7 years (i.e., until 22 years after the date of the original loan). In the negotiations the lender agrees to these terms.

\subsection*{Question 10, Section 3, Chapter 5  (\cite{toi3rd})}

\noindent A loan is being repaid with a series of payments at the end of each quarter for five years. If the amount of principal in the third payment is \$100, find the amount of principal in the last five payments. Interest is at the rate of 10\% convertible quarterly.

\subsection*{Question 12, Section 3, Chapter 5  (\cite{toi3rd})}

\noindent A borrower has a mortgage that calls for level annual payments of 1 at the end of each year for 20 years. At the time of the seventh regular payment an additional payment is made equal to the amount of principal that, according to the original amortization schedule, would have been repaid by the eighth regular payment. If payments of 1 continue to be made at the end of the eighth and succeeding years until the mortgage is fully repaid, show that the amount saved in interest payments over the full term of the mortgage is
\[
1 - v^{13}.
\]

\subsection*{Question 14, Section 3, Chapter 5  (\cite{toi3rd})}

\noindent A 35-year loan is to be repaid with equal installments at the end of each year. The amount of interest paid in the 8th installment is \$135. The amount of interest paid in the 22nd installment is \$108. Calculate the amount of interest paid in the 29th installment.

\subsection*{Question 16, Section 3, Chapter 5  (\cite{toi3rd})}

\noindent A bank customer borrows \( X \) at an annual effective rate of 12.5\% and makes level payments at the end of each year for \( n \) years.

\begin{enumerate}
\item[(i)] The interest portion of the final payment is \$153.86.
\item[(ii)] The total principal repaid as of time \( n-1 \) is \$6009.12.
\item[(iii)] The principal repaid in the first payment is \( Y \).
\end{enumerate}
Calculate \( Y. \)

\subsection*{Question 17, Section 4, Chapter 5  (\cite{toi3rd})}

\noindent A has borrowed \$10{,}000 on which interest is charged at 10\% effective. A is accumulating a sinking fund at 8\% effective to repay the loan. At the end of 10 years the balance in the sinking fund is \$5000. At the end of the 11th year A makes a total payment of \$1500.

\begin{enumerate}
\item[(a)] How much of the \$1500 pays interest currently on the loan?  
\item[(b)] How much of the \$1500 goes into the sinking fund?  
\item[(c)] How much of the \$1500 should be considered as interest?  
\item[(d)] How much of the \$1500 should be considered as principal?  
\item[(e)] What is the sinking fund balance at the end of the 11th year?  
\end{enumerate}

\subsection*{Question 21 Section 4, Chapter 5  (\cite{toi3rd})}

\noindent A borrows \$12{,}000 for 10 years and agrees to make semiannual payments of \$1000. The lender receives 12\% convertible semiannually on the investment each year for the first 5 years and 10\% convertible semiannually for the second 5 years. The balance of each payment is invested in a sinking fund earning 8\% convertible semiannually.  Find the amount by which the sinking fund is short of repaying the loan at the end of the 10 years. Answer to the nearest dollar.

\subsection*{Question 22, Section 4, Chapter 5  (\cite{toi3rd})}

\textbf{(a)} A borrower takes out a loan of \$3000 for 10 years at 8\% convertible semiannually.  
The borrower replaces one third of the principal in a sinking fund earning 5\% convertible semiannually and the other two thirds in a sinking fund earning 7\% convertible semiannually.  
Find the total semiannual payment.

\textbf{(b)} Rework (a) if the borrower each year puts one third of the total sinking fund deposit into the 5\% sinking fund and the other two thirds into the 7\% sinking fund.

\textbf{(c)} Justify from general reasoning the relative magnitude of the answers to (a) and (b).

\subsection*{Question 23, Section 4, Chapter 5  (\cite{toi3rd})}

\noindent A payment of \$36{,}000 is made at the end of each year for 31 years to repay a loan of \$400{,}000. If the borrower replaces the capital by means of a sinking fund earning 3\% effective, find the effective rate paid to the lender on the loan.

\subsection*{Question 24, Section 4, Chapter 5  (\cite{toi3rd})}

\noindent A borrows \$1000 for 10 years at an annual effective interest rate of 10\%. A can repay this loan using the amortization method with payments of \( P \) at the end of each year. Instead, A repays the loan using a sinking fund that pays an annual effective rate of 14\%. The deposits to the sinking fund are equal to \( P \) minus the interest on the loan and are made at the end of each year for 10 years. Determine the balance in the sinking fund immediately after repayment of the loan.

\newpage

\section*{Solutions}

\subsection*{Question 4, Section 2, Chapter 4  (\cite{toi3rd})}

\noindent An annuity-immediate that pays 400 quarterly for the next 10 years costs \$10{,}000. Calculate the nominal interest rate convertible monthly earned by this investment.

\[
10000 = 400 a_{\angl{40}}^{(4)} = 400 \frac{1 - (1 + i_q)^{-40}}{i_q}
\]
\[
25 = \frac{1 - (1 + i_q)^{-40}}{i_q}
\]
Solving numerically gives \( i_q \approx 0.010386 \Rightarrow i^{(4)} = 0.041544 \).

\[
i_{\text{eff}} = (1 + i_q)^4 - 1 = 0.0424
\]
\[
(1 + i_{\text{eff}}) = (1 + i^{(12)}/12)^{12}
\Rightarrow i^{(12)} = 12[(1.0424)^{1/12} - 1] = 0.04155
\]
\[
\boxed{i^{(12)} = 4.16\%}
\]

\subsection*{Question 7, Section 3, Chapter 4  (\cite{toi3rd})}

\noindent Find an expression for the present value of an annuity-due of \$600 per annum payable semiannually for 10 years if \( d^{(12)} = 0.09. \)

\[
d^{(12)} = 0.09 \Rightarrow d_m = 0.0075
\]
\[
v = 1 - d_m = 0.9925
\]
\[
d^{(2)} = 1 - (1 - d_m)^6, \quad i^{(2)} = \frac{d^{(2)}}{1 - d^{(2)}}
\]
\[
PV = 300 (1 + i^{(2)}) a_{\angl{20}}^{(2)} = 300 (1 + i^{(2)}) \frac{1 - (1 + i^{(2)})^{-20}}{i^{(2)}}
\]
\[
\boxed{PV = 300(1 + i^{(2)}) \frac{1 - (1 + i^{(2)})^{-20}}{i^{(2)}}}
\]

\subsection*{Question 9, Section 3, Chapter 4  (\cite{toi3rd})}

\noindent Find an expression for the present value of an annuity on which payments are \$100 per quarter for five years, just before the first payment is made, if \( \delta = 0.08. \)

\[
v = e^{-\delta/4}, \quad i_q = e^{\delta/4} - 1
\]
\[
a_{\angl{20}} = \frac{1 - v^{20}}{i_q}
\]
\[
PV = 100 a_{\angl{20}} = 100 \frac{1 - e^{-5\delta}}{e^{\delta/4} - 1}
\]
\[
\boxed{PV = 100 \frac{1 - e^{-5\delta}}{e^{\delta/4} - 1}}
\]

\subsection*{Question 10, Section 3, Chapter 4  (\cite{toi3rd})}

\noindent Find an expression for the present value of an annuity on which payments are 1 at the beginning of each 4-month period for 12 years, assuming a rate of interest per 3-month period.

\[
\text{Each 4-month period } = \frac{4}{3} \text{ quarters}, \quad v = (1 + i)^{-4/3}
\]
Number of payments: \( n = \frac{12 \times 12}{4} = 36 \).
\[
PV = (1 + i)^{4/3} \frac{1 - (1 + i)^{-36 \times 4/3}}{(1 + i)^{4/3} - 1}
\]
\[
\boxed{PV = (1 + i)^{4/3} \frac{1 - (1 + i)^{-48}}{(1 + i)^{4/3} - 1}}
\]

\subsection*{Question 13, Section 4, Chapter 4  (\cite{toi3rd})}

\noindent A sum of \$10{,}000 is used to buy a deferred perpetuity-due paying \$500 every six months forever. Find an expression for the deferred period expressed as a function of \( d. \)

\[
10000 = \frac{500 (1 + (1 - d)^{1/2})(1 - d)^{2k}}{d/2}
\]
\[
(1 - d)^{2k} = \frac{10d}{1 + \sqrt{1 - d}}
\]
\[
\boxed{k = \frac{1}{2} \frac{\ln\!\left(\frac{10d}{1 + \sqrt{1 - d}}\right)}{\ln(1 - d)}}
\]

\subsection*{Question 14, Section 4, Chapter 4  (\cite{toi3rd})}

\noindent If \( 3 a_{\angl{n}}^{(2)} = 2 a_{\angl{2n}}^{(2)} = 45 s_{\angl{n}}^{(2)} \), find \( i \).

\[
3(1 - v^n) = 2(1 - v^{2n}) \Rightarrow 1 = 3v^n - 2v^{2n}
\]
\[
2v^{2n} - 3v^n + 1 = 0 \Rightarrow v^n = \frac{1}{2}
\]
\[
(1 + i)^n = 2 \Rightarrow \boxed{i = 2^{1/n} - 1}
\]


\subsection*{Question 17, Section 5, Chapter 4  (\cite{toi3rd})}

\noindent There is \$40{,}000 in a fund which is accumulating at 4\% per annum convertible continuously. If money is withdrawn continuously at the rate of \$2{,}400 per annum, how long will the fund last?

\[
40000 = \frac{2400}{0.04}(e^{0.04t} - 1)
\]
\[
e^{0.04t} = 1 + \frac{40000(0.04)}{2400} = 1.6667
\]
\[
t = \frac{\ln(1.6667)}{0.04} = 12.8
\]
\[
\boxed{t = 12.8 \text{ years}}
\]

\subsection*{Question 19, Section 5, Chapter 4  (\cite{toi3rd})}

\noindent Find an expression for \( \bar{a}_{\angl{n}} \) if \( \delta_t = \frac{1}{1 + t} \) for \( 0 \le t \le n. \)

Find \( \bar{a}_{\angl{n}} \) if \( \delta_t = \frac{1}{1 + t} \).

\[
v(t) = e^{-\int_0^t \delta_s ds} = e^{-\ln(1 + t)} = \frac{1}{1 + t}
\]
\[
\bar{a}_{\angl{n}} = \int_0^n v(t)\,dt = \int_0^n \frac{1}{1 + t}\,dt = \ln(1 + n)
\]
\[
\boxed{\bar{a}_{\angl{n}} = \ln(1 + n)}
\]

\subsection*{Question 22, Section 6, Chapter 4  (\cite{toi3rd})}

\noindent Simplify
\[
\sum_{t=1}^{20} (t + 5)v^t.
\]

\[
\sum_{t=1}^{20} (t + 5)v^t = \sum_{t=1}^{20} t v^t + 5 \sum_{t=1}^{20} v^t
\]
\[
\sum_{t=1}^{n} v^t = \frac{v(1 - v^n)}{1 - v}, \quad
\sum_{t=1}^{n} t v^t = \frac{v(1 - (n+1)v^n + n v^{n+1})}{(1 - v)^2}
\]
\[
\boxed{\sum_{t=1}^{20} (t + 5)v^t =
\frac{v(1 - 21v^{20} + 20v^{21})}{(1 - v)^2}
+ 5\frac{v(1 - v^{20})}{1 - v}}
\]

\subsection*{Question 27, Section 6, Chapter 4  (\cite{toi3rd})}

\noindent An annuity-immediate has semiannual payments of 800, 750, 700, \dots, 350, at \( i^{(2)} = 0.16. \)  
If \( a_{\angl{10}, 0.08} = A \), find the present value of the annuity in terms of \( A. \)

\[
\text{Payments: } a_1 = 800, d = -50
\]
\[
PV = \sum_{k=1}^{10} (800 - 50(k-1))v^k
\]
\[
PV = 800 a_{\angl{10}} - 50 \sum_{k=1}^{10}(k-1)v^k
\]
\[
\sum_{k=1}^{10}(k-1)v^k = \frac{v(1 - 10v^9 + 9v^{10})}{(1 - v)^2}
\]
\[
\boxed{PV = 800A - 50 \frac{v(1 - 10v^9 + 9v^{10})}{(1 - v)^2}}
\]

\subsection*{Question 30, Section 7, Chapter 4  (\cite{toi3rd})}

Annual deposits are made at the beginning of each year for 10 years.  
The first 5 deposits are \$1000 each, and thereafter they increase by 5\% per year.  
The fund earns 8\% effective.

\[
\text{First 5 deposits: } 1000 \ddot{s}_{\angl{5}|0.08}
\]
\[
\text{Remaining 5 deposits: } 1000(1.05)^5 \ddot{s}_{\angl{5}|0.08, g=0.05}
\]
For the second part, the geometric accumulation factor applies only to years 6–10:
\[
AV_{10} = 1000(1.08)^5 \left[\frac{(1.08/1.05)^5 - 1}{(1.08/1.05) - 1}\right] (1.05)^5 + 1000 \frac{(1.08)^5 - 1}{0.08}
\]
Simplify numerically to find \( AV_{10} \approx 15{,}710 \).
\[
\boxed{AV_{10} = 15{,}710 \text{ (approx.)}}
\]

\subsection*{Question 31, Section 7, Chapter 4  (\cite{toi3rd})}

A perpetuity begins 5 years from now with first payment \$1000 increasing by \(k\%\).  
\(i = 25\%\), PV = 4096.

\[
PV = 1000 v^4 \frac{1}{i - g}
\]
where \(v = \frac{1}{1.25}\), \(g = \frac{k}{100}\).

\[
4096 = 1000 (1.25)^{-4} \frac{1}{0.25 - g}
\Rightarrow 0.25 - g = \frac{1000 (1.25)^{-4}}{4096}
\]
\[
g = 0.25 - \frac{0.1000}{4096} (1.25)^{-4} \approx 0.05
\]
\[
\boxed{k = 5\%}
\]

\subsection*{Question 32, Section 7, Chapter 4  (\cite{toi3rd})}

Employee aged 40, salary \$40,000, raises 3\%, contributes 4\% of salary annually at beginning for 25 years.  
Interest 5\%.

Contribution each year: \( 0.04 \times 40000 (1.03)^{t-1} \), \(t=1,\dots,25\).

\[
AV = 0.04(40000)(1.05)^{25} \frac{1 - \left(\frac{1.03}{1.05}\right)^{25}}{1 - \frac{1.03}{1.05}}
\]
\[
\boxed{AV = 1600(1.05)^{25} \frac{1 - (1.03/1.05)^{25}}{1 - (1.03/1.05)}}
\]
Numerically, \( AV \approx 84{,}700 \).
\[
\boxed{AV = 84{,}700 \text{ (approx.)}}
\]

\subsection*{Question 33, Section 7, Chapter 4  (\cite{toi3rd})}

Payments at beginning of each year for 20 years.  
First payment \$100, increasing 5\% per year to year 10, then decreasing 5\% per year afterward.  
\(i = 7\%\).

\[
PV = 100 \left[\sum_{t=0}^{9} (1.05)^t v^t + (1.05)^{10} \sum_{t=10}^{19} (0.95)^{t-10} v^t \right] (1+i)
\]
Simplify:
\[
PV = 100(1.07)\left[\frac{1 - (1.05v)^{10}}{1 - 1.05v} + (1.05)^{10} v^{10} \frac{1 - (0.95v)^{10}}{1 - 0.95v}\right]
\]
\[
\boxed{PV = 100(1.07)\left[\frac{1 - (1.05v)^{10}}{1 - 1.05v} + (1.05)^{10} v^{10} \frac{1 - (0.95v)^{10}}{1 - 0.95v}\right]}
\]

\subsection*{Question 34, Section 8, Chapter 4  (\cite{toi3rd})}

\[
(I^{(m)} a_{\angl{n}})^{(m)} = \frac{1}{m^2}\left[\nu^m + 2\nu^{2m} + \cdots + n m \nu^{n m}\right]
\]
\[
= \frac{\ddot{a}_{\angl{n}}^{(m)} - n\nu^n}{i^{(m)}}
\]
Derived by differentiating the accumulated value series and dividing by \(i^{(m)}\).
\[
\boxed{(I^{(m)} a_{\angl{n}})^{(m)} = \frac{\ddot{a}_{\angl{n}}^{(m)} - n\nu^n}{i^{(m)}}}
\]

\subsection*{Question 36, Section 8, Chapter 4  (\cite{toi3rd})}

Payments: 1 at end of 5th, 6th; 2 at 7th, 8th; 3 at 9th, 10th, etc.

Group pairs of payments as annuities with payment increases of 1 every 2 years.

Let \(a = v^4\). Then:
\[
PV = v^4 [1 + 2v^2 + 3v^4 + \dots] = v^4 \sum_{k=1}^{\infty} k (v^2)^{k-1} v^2
\]
\[
= \frac{v^4}{i - v d}
\]
\[
\boxed{PV = \frac{v^4}{i - v d}}
\]

\subsection*{Question 38, Section 8, Chapter 4  (\cite{toi3rd})}

A perpetuity with payments every six months starting today, first payment = 1, increasing 3\% each time.  
Interest = 8\% per annum.

Semiannual effective rate:
\[
i_{(2)} = (1.08)^{1/2} - 1 = 0.03923
\]
Growth rate per half-year \( g = (1.03)^{1/2} - 1 = 0.01489 \).

Since it’s due:
\[
PV = (1 + i_{(2)}) \frac{1}{1 - (1 + g)/(1 + i_{(2)})}
\]
\[
\boxed{PV = (1 + i_{(2)}) \frac{1}{1 - \frac{1.03^{1/2}}{1.08^{1/2}}}}
\]

\subsection*{Question 42, Section 9, Chapter 4  (\cite{toi3rd})}

\textbf{(a)}  
Integral form:
\[
(\bar{D}\bar{a})_{\angl{n}} = \int_0^n t v(t) \, dt
\]
\textbf{(b)}  
By integration by parts:
\[
(\bar{D}\bar{a})_{\angl{n}} = n \bar{a}_{\angl{n}} - \bar{s}_{\angl{n}}
\]
\[
\boxed{
(\bar{D}\bar{a})_{\angl{n}} = \int_0^n t v(t) \, dt 
\quad \text{and} \quad
(\bar{D}\bar{a})_{\angl{n}} = n\bar{a}_{\angl{n}} - \bar{s}_{\angl{n}}
}
\]

\subsection*{Question 43, Section 9, Chapter 4  (\cite{toi3rd})}

One-year deferred continuous varying annuity, payments \( t^2 - 1 \) for 13 years,  
force of interest \( \delta_t = (1 + t)^{-1} \).

Discount factor:
\[
v(t) = e^{-\int_0^t (1 + s)^{-1} ds} = \frac{1}{1 + t}
\]
Present value:
\[
PV = \int_1^{14} (t^2 - 1)v(t)\,dt = \int_1^{14} \frac{t^2 - 1}{1 + t} dt
\]
Simplify:
\[
\frac{t^2 - 1}{1 + t} = t - 1
\]
\[
PV = \int_1^{14} (t - 1)dt = \frac{1}{2}(t - 1)^2 \bigg|_1^{14} = \frac{1}{2}(13)^2 = 84.5
\]
\[
\boxed{PV = 84.5}
\]

\subsection*{Question 46, Section 9, Chapter 4  (\cite{toi3rd})}

An annuity pays \$100 monthly for 9 months each year for 4 years.  
Payments are made at end of each month.

Equivalent to \$100 paid monthly for 4 years, but only for 9 out of 12 months each year.

Total payments each year: \( 9 \times 100 = 900 \),  
effective monthly interest \( i^{(12)}/12 \).

\[
PV = 100 \times 9 \, a_{\angl{4}}^{(12)} q_{9/12}^{(12)} = 1200 \ddot{a}_{\angl{4}} q_{9/12}^{(12)}
\]
\[
\boxed{PV = 1200 \ddot{a}_{\angl{4}} q_{9/12}^{(12)}}
\]

\subsection*{Question 6, Section 2, Chapter 5  (\cite{toi3rd})}

Let the regular annual payment be \( R \).  
The outstanding balance after 20 payments is:
\[
R a_{\angl{5}} = R(a_{\angl{25}} - a_{\angl{20}}).
\]
To retire the debt in 20 years instead of 25, additional payments of \( K \) are made with the 6th through 10th payments.  
Thus:
\[
K a_{\angl{5}} = R(a_{\angl{25}} - a_{\angl{20}}).
\]
Hence,
\[
K = \frac{a_{\angl{25}} - a_{\angl{20}}}{a_{\angl{5}}}.
\]

\subsection*{Question 7, Section 2, Chapter 5  (\cite{toi3rd})}

The effective rate is \( i = 0.065 \).

The annual payment for a 15-year, \$150{,}000 mortgage is:
\[
R = \frac{150{,}000}{a_{\angl{15}|0.065}} = 15{,}952.92.
\]
After 5 years, the outstanding balance is:
\[
B_5 = R a_{\angl{10}|0.065} = 114{,}682.82.
\]
They borrow an additional \$80{,}000, extending the loan to 22 years (17 more years).  
New balance:
\[
114{,}682.82 + 80{,}000 = 194{,}682.82 = R_2 a_{\angl{17}|0.065}.
\]
Thus,
\[
R_2 = 19{,}255.36.
\]

\subsection*{Question 10, Section 3, Chapter 5  (\cite{toi3rd})}

Quarterly rate \( i_q = 0.10 / 4 = 0.025. \)

The principal portion in the \(k\)-th payment is:
\[
P_k = P_3 (1 + i_q)^{k-3}.
\]
Given \( P_3 = 100 \), the total principal in the last five payments is:
\[
\sum_{k=16}^{20} 100(1.025)^{k-3} = 724.59.
\]

\subsection*{Question 12, Section 3, Chapter 5  (\cite{toi3rd})}

The borrower makes level annual payments of 1 for 20 years.

At time 7, an extra payment equal to the principal of the 8th payment is made.  
If payments of 1 continue as scheduled, the total interest saved is the difference between the interest that would have been paid over 13 remaining years:
\[
\text{Interest saved} = 1 - v^{13}.
\]

\subsection*{Question 14, Section 3, Chapter 5  (\cite{toi3rd})}

Given:
\[
n = 35, \quad I_8 = 135, \quad I_{22} = 108.
\]
The interest in the \(k\)-th payment is proportional to the outstanding balance:
\[
I_k = i L (1 + i)^{n - k + 1} / a_{\angl{n}|i}.
\]
Hence:
\[
\frac{I_8}{I_{22}} = (1 + i)^{14}.
\]
From \(135 / 108 = (1 + i)^{14}\), we find \(i = 0.1041.\)

Then the interest in the 29th payment:
\[
I_{29} = 135(1.1041)^{-21} = 72.0.
\]

\subsection*{Question 16, Section 3, Chapter 5  (\cite{toi3rd})}

Effective rate \( i = 0.125 \).

(i) \( I_n = i(X - P_{n-1}) = 153.86 \Rightarrow X - P_{n-1} = \frac{153.86}{0.125} = 1230.88. \)

(ii) Total principal repaid as of time \(n-1\) is \( 6009.12 \), so:
\[
X = 6009.12 + 1230.88 = 7240.00.
\]

(iii) Payment \( R = \frac{X}{a_{\angl{n}|i}}. \)
Given \(a_{\angl{n}|i} = 5.2284\), \(R = 1385.13.\)
First principal repayment:
\[
Y = R - iX = 1385.13 - 0.125(7240) = 479.73.
\]

\subsection*{Question 17, Section 4, Chapter 5  (\cite{toi3rd})}

Loan interest rate \( i_L = 10\% \), sinking fund rate \( i_S = 8\%. \)

(a) Interest on loan: \(10{,}000 \times 0.10 = 1{,}000.\)

(b) Amount going to sinking fund: \(1{,}500 - 1{,}000 = 500.\)

(c) Interest earned in the fund: \(5{,}000 \times 0.08 = 400.\)

(d) Interest considered for the year: \(1{,}000 - 400 = 600.\)

(e) Fund balance at end of year 11:
\[
5{,}000(1.08) + 500 = 5{,}900.
\]

\subsection*{Question 21 Section 4, Chapter 5  (\cite{toi3rd})}

Loan: \$12{,}000, semiannual payments of \$1{,}000 for 10 years (20 periods).

First 5 years: \(j_1 = 0.06\), next 5 years: \(j_2 = 0.05\).  
Sinking fund rate \(j_f = 0.04.\)

Compute fund balance after 10 years: \(F = 9{,}778.59.\)

The shortfall:
\[
12{,}000 - 9{,}778.59 = 2{,}221.41.
\]

\subsection*{Question 22, Section 4, Chapter 5  (\cite{toi3rd})}

(a) \( L = 3{,}000, \; j = 0.04, \; n = 20. \)

Interest payment each period: \(120.\)
Deposits:
\[
\begin{aligned}
d_1 &= \frac{1{,}000}{s_{\angl{20}|0.025}} = 39.15,\\
d_2 &= \frac{2{,}000}{s_{\angl{20}|0.035}} = 70.72.
\end{aligned}
\]
Total semiannual payment:
\[
120 + 39.15 + 70.72 = 229.87.
\]

(b) With mixed sinking fund: \(D = 109.62 \Rightarrow 229.62.\)

(c) Result: total payments are almost identical; averaging the two rates reduces variation.

\subsection*{Question 23, Section 4, Chapter 5  (\cite{toi3rd})}

\[
R = 36{,}000, \quad n = 31, \quad L = 400{,}000, \quad i_S = 0.03.
\]
The effective rate \(i\) satisfies:
\[
400{,}000 = R a_{\angl{31}|i}(1 + i_S)^{31}.
\]
Solving numerically gives:
\[
i_{\text{effective}} \approx 12.36\%.
\]

\subsection*{Question 24, Section 4, Chapter 5  (\cite{toi3rd})}

\[
L = 1{,}000, \quad i = 0.10, \quad j = 0.14, \quad n = 10.
\]
Amortization payment:
\[
P = \frac{L i}{1 - (1+i)^{-n}}.
\]
Deposit each year into sinking fund:
\[
d = P - L i = P - 100.
\]
Fund balance after 10 years:
\[
S = d \, s_{\angl{10}|0.14} = (P - 100)\frac{(1.14)^{10} - 1}{0.14}.
\]


\newpage
\printbibliography
\end{document}
